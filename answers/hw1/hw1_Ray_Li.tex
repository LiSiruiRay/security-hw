\documentclass[11pt]{article}
\usepackage[margin=1in]{geometry}
\usepackage{enumitem}
\usepackage{hyperref}
\usepackage{amsmath}
\title{HW 1}
\author{Ray Li - CSCI-UA.0480-63}
\date{\today}
\begin{document}
\maketitle

\section*{1. Threat modeling for Citi Bike}
Here are three security policies for New York's Citi Bike system, each targeting a distinct adversary type (thieves, terrorists, trolls). For each policy, we specify concrete enforcement mechanisms and rationale.

\subsection*{Policy A (Thieves): Ensure bikes cannot be used or resold if unlawfully removed}
\textbf{Goal}: Reduce economic incentive by making stolen bikes valueless and hard to monetize.\\
\textbf{Solutions}:
\begin{itemize}[leftmargin=*]
  \item \textit{Tamper-resistant electronic locks with authenticated release}: Require station or on-bike lock to perform mutual authentication with backend before unlock; deny offline unlocks; sign unlock events to enable audit. Implement geo-fencing so a bike that leaves a service area without an authorized session hard-locks. (Defense-in-depth policy/mechanism separation discussed in Notes 2025.01.01.)
  \item \textit{On-bike immobilizer with cryptographic pairing}: Pair controller to per-bike keys; if controller or battery is replaced without re-provisioning via service tool, firmware refuses to engage motor/gears. (Avoid security by obscurity; rely on secret keys rather than hidden design; cf. Kerckhoffs principle, Notes 2025.01.04.)
  \item \textit{Active recovery and deterrence}: GPS/Cellular beaconing with periodic attestations; stolen-mode triggers brighter lighting patterns and server-side location beacons for recovery in collaboration with NYPD.
\end{itemize}
\textbf{Why this helps}: Removes resale utility and increases recovery likelihood, shifting attacker utility (Notes 2025.01.01 on attacker/defender utilities).

\subsection*{Policy B (Terrorists): Maintain rider safety and rapid incident response}
\textbf{Goal}: Preserve safety and availability in the face of attempts to cause violent disruption.\\
\textbf{Solutions}:
\begin{itemize}[leftmargin=*]
  \item \textit{Station and fleet integrity monitoring}: Backend analytics for simultaneous abnormal unlocks, sudden mass rebalancing anomalies, or tamper alerts indicating coordinated sabotage; automated incident playbooks to disable unlocks in affected zones.
  \item \textit{Physical hardening and surveillance at high-risk sites}: Bollards, tamper-evident fasteners, and CCTV coverage at major transit hubs to deter placement of hazards and to support forensics.
  \item \textit{Emergency broadcast and geo-fenced shutdown}: Capability to pause rentals and push safety notices within a radius; integrate with city emergency systems for coordinated response.
\end{itemize}
\textbf{Why this helps}: Defense-in-depth across detection, prevention, and response reduces likelihood and impact (Notes 2025.01.01 on defense-in-depth).

\subsection*{Policy C (Trolls): Maintain service availability and fair access}
\textbf{Goal}: Limit nuisance attacks (dock blocking, false reports, mass reservation griefing).\\
\textbf{Solutions}:
\begin{itemize}[leftmargin=*]
  \item \textit{Strong user authentication and rate-limiting}: Bind accounts to verified payment tokens; apply per-user and per-IP quotas on reservations, unlock attempts, and incident reports; require proof-of-ride (e.g., short post-unlock movement) to keep a dock reserved.
  \item \textit{Anomaly scoring and graduated friction}: Add friction (CAPTCHAs, SMS recheck) when behavior deviates from normal usage to deter automated griefing.
  \item \textit{Dock occupancy attestation}: Use on-dock sensors plus bike IMU to verify a dock is truly occupied; auto-release ghost holds.
\end{itemize}
\textbf{Why this helps}: Increases attacker cost for nuisance actions while minimizing friction for legitimate riders (Notes 2025.01.01 on policy vs. mechanism and attacker utilities).

\section*{2. Hash functions and privacy}

\subsection*{Why plain hashing phone numbers does not protect privacy}
Phone numbers live in a tiny, structured space (e.g., 10--15 digits with known national formats). An adversary can simply enumerate all plausible phone numbers, hash each, and match the server's values. This defeats confidentiality despite hash one-wayness (one-wayness resists inverting a random preimage, not exhaustive enumeration of low-entropy domains). It also leaks users' entire address books to the server via set-membership queries.


\section*{3. MACs vs. PRFs}
\subsection*{(a) PRF implies MAC (short proof)}
Assume there exists an adversary A$_1$ that wins the MAC existential unforgeability game against F with non-negligible probability. We construct A$_2$ to win the PRF distinguishing game using A$_1$ as a subroutine:
\begin{itemize}[leftmargin=*]
  \item A$_2$ receives oracle O(·) which is either F(k,·) or a truly random function R(·).
  \item A$_2$ runs A$_1$ and answers each MAC query m by returning t\,=\,O(m).
  \item When A$_1$ outputs a purported forgery (m$^*$, t$^*$) with m$^*$ not previously queried, A$_2$ outputs “real” iff O(m$^*$)=t$^*$, else “random”.
\end{itemize}
\textbf{Analysis}: If O=F(k,·) then A$_1$ sees a perfect MAC oracle and, by assumption, produces a valid new (m$^*$, t$^*$) with non-negligible probability, so A$_2$ distinguishes PRF-from-random with the same advantage. If O=R(·), then t$^*$ is independent of R(m$^*$) and matches with probability at most 2$^{-t}$ (negligible). Hence any MAC forger yields a PRF distinguisher, proving a secure PRF is also a secure MAC.

\subsection*{(b) A MAC that is not a PRF}
Let F be a secure PRF. Define F'(k, m) that outputs the pair (F(k, m), 0$^t$) (or equivalently, append a fixed trailer bit). Verification accepts a tag t' iff its first t bits equal F(k, m). Then:
\begin{itemize}[leftmargin=*]
  \item \textbf{MAC security}: Given oracle access, producing a fresh (m, t') requires predicting F(k, m) on a new m, which is hard if F is a PRF. The fixed trailer does not help forgery.
  \item \textbf{Not a PRF}: F' is trivially distinguishable from random because its output always ends with 0$^t$. A random function produces that pattern with probability 2$^{-t}$, yielding a huge distinguishing advantage.
\end{itemize}

\section*{4. Semantic security for the Vigen\`{e}re cipher}
The Vigen\`{e}re cipher with a fixed period p=10 is not semantically secure. We give a no-query adversary that wins the IND-CPA game with probability 1.

\subsection*{Attack idea}
\begin{enumerate}[leftmargin=*]
  \item Choose two equal-length plaintexts m0, m1 longer than 10 so that for some position i, the letters at all indices congruent to i mod 10 differ between m0 and m1 but are identical within each plaintext across those positions. For example, for English letters let the positions i, i+10, i+20, ... in m0 all be 'A' and in m1 all be 'B'; keep all other positions equal between m0 and m1.
  \item Submit m0, m1 as the challenge messages. Let the challenger encrypt mb under a uniformly random 10-letter key k and return c.
  \item Inspect ciphertext letters at positions i, i+10, i+20, ...: under Vigen\`{e}re, each of these positions is a shift by the same key letter k[i]. Because the plaintext letters at those positions are constant within a message, the corresponding ciphertext letters will also be constant (but shifted) within each message. Compute the shift between c[i] and c[i+10]. If they are equal shifts for 'A', output b=0; if they align to the shift for 'B', output b=1.
\end{enumerate}
This succeeds with probability 1 since the repeated key letter induces identical shifts at every i+10j position, letting the adversary tell which constant letter ('A' vs 'B') was encrypted.

\subsection*{Why this breaks semantic security}
IND-CPA requires that no efficient adversary can distinguish encryptions of chosen equal-length messages with probability non-negligibly greater than 1/2. Here, the periodic key leaks class information (the congruence class modulo 10), enabling frequency/structure-based distinguishing on repeated positions, violating semantic security. The weakness is exactly the key-reuse periodicity. In contrast, a one-time pad (non-repeating key) would be semantically secure, and modern randomized modes ensure semantic security by using nonces/IVs.

\section*{5. Block cipher modes of operation: plaintext block chaining (PBC)}
Let PBC be defined by $c_0 = E_K(m_0) \oplus IV$ and for $i\ge 1$: $c_i = E_K(m_i) \oplus m_{i-1}$. This lets the sender evaluate all $E_K(m_i)$ in parallel, then xor with the known previous plaintext (or IV for the first block).

\subsection*{Attack when $m_1 = m_2 = x$ (known)}
Given public $IV, c_0, c_1, c_2$ and a known block $x$ with $m_1=m_2=x$:\
1) From $c_2 = E_K(m_2) \oplus m_1 = E_K(x) \oplus x$, compute $E_K(x) = c_2 \oplus x$.

2) From $c_1 = E_K(m_1) \oplus m_0 = E_K(x) \oplus m_0$, recover the first block:
\[ m_0 = E_K(x) \oplus c_1 = (c_2 \oplus x) \oplus c_1. \]
This uses only public values and the known $x$. Hence PBC is not semantically secure: repetitions of plaintext blocks leak $E_K(x)$ and reveal neighboring plaintexts.

\section*{6. CBC-MAC}
Alice computes a MAC by running CBC encryption and keeping only the final block (tag). Let the block cipher be $E_K$, IV be public, and messages be multiples of the block size: for a message with blocks $(m_0,\dots,m_{\ell-1})$:
\[ c_{-1} = IV,\quad c_i = E_K(m_i \oplus c_{i-1}),\quad \text{Tag}(M)=c_{\ell-1}. \]
This scheme is insecure if the IV is not fixed to zero and/or message lengths are not fixed. We show existential forgeries using only observed tags.

\subsection*{Forgery from two known (IV, Tag) pairs}
Suppose we know $(IV_1, T_1)$ for message $M_1=(m^{(1)}_0,\dots,m^{(1)}_{a-1})$ and $(IV_2, T_2)$ for message $M_2=(m^{(2)}_0,\dots,m^{(2)}_{b-1})$. Consider the crafted message
\[ M_3 = M_1 \parallel (m^{(2)}_0 \oplus IV_2 \oplus T_1) \parallel m^{(2)}_1 \parallel \cdots \parallel m^{(2)}_{b-1}. \]
Then CBC chaining over $M_3$ with IV$=IV_1$ yields $c_{a-1}=T_1$ and the next block input equals $m^{(2)}_0\oplus IV_2$, so the remainder of the computation exactly reproduces the path that produced $T_2$. Hence
\[ \text{Tag}(M_3) = T_2, \qquad \text{using } (IV_1,T_2). \]
This is a valid existential forgery unless $M_3$ was previously MACed.

\subsection*{One-block forgery (when a single-block tag is known)}
If an observed message $M$ has a single block $x$ with pair $(IV,T)$, then $T=E_K(x\oplus IV)$. For any chosen block $y$, define
\[ M' = x \parallel (y \oplus IV \oplus T). \]
Its tag under IV is $E_K(y \oplus IV)$, i.e., the tag that would be produced for the one-block message $y$. Thus we can forge a valid tag for $y$ without the key.

\subsection*{Why this fails}
CBC-MAC is only secure when the IV is fixed (often zero) and the message length is fixed and included in the computation; otherwise, the chaining malleability above enables splice-and-extend forgeries.

\end{document}
